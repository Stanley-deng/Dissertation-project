\chapter{Result discussion}
In the previous chapter, the tool is put into test against influenza1918 and Covasim, most result shows that the tool is capable of assisting user in the testing process. But there are still parts of it that can be enhanced.
\section{A summary of results}
In the result provided by testing on influenza1918, it shows that with the help of the tool, testing become more convenient. The process is overall simplified by reduce the need for user to directly interact with the coding part of testing. With the parts that is necessary for the process, it is mostly reduced by auto generate most of it, with some minor parts still require to be define. This is also the same when defining \textsl{.feature} file, without the need to deal with syntax and format, this process become a lot more smoother. \\*\\*
For covasim, the experience is mostly same when it comes defining relations for parameter and creating feature file, without the need to directly code these files, and with the help of the GUI, the testing process is a lot more understandable and streamline. But when it comes to the background files, despite the auto generated parts, there are still more required to be define. Consider that Covasim is a more complex model, this is to be expected, but this part also shows that if the user tends to add a lot of custom testing aspect, the testing process could still become lengthy.\\*\\*
Combine the information provide by two models, it shows that in terms of editing \textsl{.feature} file and \textsl{.dot} file. This tool can effectively assist user, but when it comes to the background file, if the user is intended to have many custom conditions, then the user will be required to do a decent amount of editing in the background files, this may discourage people to participate in testing.
\section{How does the tool help user simplify the testing process}
