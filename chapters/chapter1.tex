\chapter{Introduction}

Throughout the years, companies have developed CPU with more and more powerful processing power, and these more powerful CPU comes the invention of supercomputer, these supercomputers have superior computational power compared to normal computers. With this computational power, people start to develop large and complex computational models. Computational model is a way to try and use computer system to simulate real life situation, these models use large amounts of parameters to determine the outcomes of a certain event, researchers can utilize these models to simulate experiments and real-life event \cite{Reference1} and aid their research with the results.\\*\\*
But these models aren’t without problems, one problem is these models has a large number of inputs and outputs, and it takes a lot of times to process them, this means in order to test them, it requires a lot of time to test all inputs and outputs’ accuracy with traditional method. \\*\\*
The solution provided by this project is to use the causal models, by determine the relationship between different input and output parameters and use this information to create graphs that focus on certain behaviours. Then just simply compare the difference between the test sets and the output of the tested models, people can know how accurate a certain interaction is between input and output parameters, then with these result, testers can decide whether or not this system meets their expectations. By using this method tester can saved a lot of time compared to traditional method where the best ways to test it is to run the model repeatedly with new data \cite{Reference2}.   


\section{Aims and Objectives}

Despite the existence of the theory, there isn’t a software where the user can store these test sets and used these sets to test other tools. So, primary goal of this project is to provide a system that can be used to store the test sets, and when needed, people can specify the scenarios they want and use the stored sets to test other computational models. \\*\\*
This system will be primary based on a tool called Causcumber, it is a tool for testing computational model and is mainly based on a model called Covasim, Covasim is a computational model focus on COVID-19 analyses, it can determine the infection rate and how will the rate change based on different interventions (such as quarantine, social distance, etc.) \cite{Reference3}. Causcumber is a in developing tool and is aimed to provide a testing method for computational model like Covasim, it uses Cucumber specification, another tool reads executable specification in plain text and validates if the software does what those specifications say, to produce casual model graph. \\*\\*
The main problem with Causcumber is that it lacks an accessible user interface, so for users who haven’t really studied software engineering, it can be difficult to understand. So, the goal is to eventually provide a user interface that can be accessible for all levels of users. \\*\\*
Another goal of this system is that it needs to be easy to read, and to achieve this, we use gherkin reference to help with it, by using a set of special keywords with a certain structure, it can give meaning to executable specifications. For the users who are not specialize in software engineering, this is a way to make the system easy to understand and can avoid a lot of confusion.


\section{Overview of the Report}

In the next chapter, Literature Survey, a list of literature will be providing information and reasons why there’s a need for this type of testing tool to exist. It will start with explaining what scientific software is, what they do and why they aren’t usually tested in a proper method. Then this will fellow up with introductions to Cucumber and Behave, two of the main factors of the Causcumber system. After that will be explain what is Causal testing and how computational model testing could benefit from it.\\* \\* 
In the third chapter, Requirements and Analysis, will be mainly talking about the requirements and objectives of the system. And the design choice was made based on these requirements. \\* \\*
Part four, Progress, will be discussing the current state of the system, how it is received and what feedback it gets. \\* \\*
In the fifth chapter, conclusions and project plan, will summarize what has been achieved till this date, and the overall plan for the upcoming months.






