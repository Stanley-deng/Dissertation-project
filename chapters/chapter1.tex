\chapter{Introduction}

With the invent of CPU with powerful processing power and the invention of supercomputer, large and complex computational models are invented and used by many corporations and government departments to determine the outcome of a certain event, these models use large number of input parameters and an algorithm to calculate the result.
But the problem with these models is due to its large sets of parameters, it could take a lot of time to process and test its’ accuracy with traditional method. This project’s solution to this is to use the causal models, by determine how an input parameter can affect the output parameters, we can use these data to create graphs that focus on certain behaviours. Then with these recorded behaviours, we can use it to test other computational models with similar input parameters by just simply compare the difference between the test sets and the output of the tested models, and by doing so we can save a lot of time.

\section{Aims and Objectives}

Despite the existence of the theory, there isn’t a software where the user can store these test sets and used these sets to test other tools. The goal of this project is to develop a system that can be used to store the test sets, and when needed, people can specify the scenarios they want and use the stored result to test other computational models. 
This system needs to be easy to read, so we decide to use the gherkin reference to help with it, by using a set of special keywords, we can give structure and meaning to executable specifications. For the users who are not specialize in this area, this makes the system easy to understand and can avoid a lot of confusion. 
For this project, we have will be develop a tool based on the Covasim model. Covasim is a computational model focus on COVID-19 analyses, it can determine the infection rate and how will the rate change based on different interventions (such as quarantine, social distance, etc.). 
A basic tool for this project has already been developed call Causcumber, this is a tool where it uses Cucumber specification, another tool reads executable specification in plain text and validates if the software does what those specifications say, to produce causal model graphs for the Covasim, and with the result we can use to test the accuracy of the Covasim model. 

\section{Overview of the Report}

Lorem ipsum dolor sit amet, consectetuer adipiscing elit. Aenean commodo ligula eget dolor. Aenean massa. Cum sociis natoque penatibus et magnis dis parturient montes, nascetur ridiculus mus. Donec quam felis, ultricies nec, pellentesque eu, pretium quis, sem. Nulla consequat massa quis enim. Donec pede justo, fringilla vel, aliquet nec, vulputate eget, arcu. In enim justo, rhoncus ut, imperdiet a, venenatis vitae, justo. Nullam dictum felis eu pede mollis pretium. Integer tincidunt. Cras dapibus. Vivamus elementum semper nisi. Aenean vulputate eleifend tellus. Aenean leo ligula, porttitor eu, consequat vitae, eleifend ac, enim. Aliquam lorem ante, dapibus in, viverra quis, feugiat a, tellus. Phasellus viverra nulla ut metus varius laoreet. Quisque rutrum. Aenean imperdiet. Etiam ultricies nisi vel augue. Curabitur ullamcorper ultricies nisi. Nam eget dui. Etiam rhoncus. Maecenas tempus, tellus eget condimentum rhoncus, sem quam semper libero, sit amet adipiscing sem neque sed ipsum. Nam quam nunc, blandit vel, luctus pulvinar, hendrerit id, lorem. Maecenas nec odio et ante tincidunt tempus. Donec vitae sapien ut libero venenatis faucibus. Nullam quis ante. Etiam sit amet orci eget eros faucibus tincidunt. Duis leo. Sed fringilla mauris sit amet nibh. Donec sodales sagittis magna. Sed consequat, leo eget bibendum sodales, augue velit cursus nunc.
