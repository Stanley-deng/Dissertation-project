\chapter{Introduction}

Throughout the years, companies have developed CPU with more and more powerful processing power, and these more powerful CPU comes the invention of supercomputer, these supercomputers have stronger computational power compared to normal computer. With this computational power, people start to develop large and complex computational models. Computational model is a way to try and use computer system to simulate real life situation, these models use large amounts of parameters to determine the outcomes of a certain event, researchers can utilize these models to simulate experiments and real-life event [1] and aid their research with the results. 

But these models aren’t without problems, one problem is these models has a large number of inputs and outputs, and it takes a lot of times to process them, this means in order to test them, it requires a lot of time to test all inputs and outputs’ accuracy with traditional method. 
The solution provided by this project is to use the causal models, by determine the relationship between different input and output parameters and use this information to create graphs that focus on certain behaviours. Then just simply compare the difference between the test sets and the output of the tested models, people can know how accurate a certain interaction is between input and output parameters, and by doing so saved a lot of time compared to traditional method where the best ways to test it is to run the model repeatedly with new data [2]. Then with these result, researchers or people who tests the model can decide whether or not this system meets their expectation.  


\section{Aims and Objectives}

Despite the existence of the theory, there isn’t a software where the user can store these test sets and used these sets to test other tools. So, the goal of this project is to develop a system that can be used to store the test sets, and when needed, people can specify the scenarios they want and use the stored result to test other computational models. 
One of the goal of this system is that it needs to be easy to read, and in order to achieve this, we use the gherkin reference to help with it, by using a set of special keywords with a certain structure, we can give meaning to executable specifications. 
//////////////////For the users who are not specialize in this area, this makes the system easy to understand and can avoid a lot of confusion. 
For this project, we have will be develop a tool based on the Covasim model. Covasim is a computational model focus on COVID-19 analyses, it can determine the infection rate and how will the rate change based on different interventions (such as quarantine, social distance, etc.). 
A basic tool for this project has already been developed call Causcumber, this is a tool where it uses Cucumber specification, another tool reads executable specification in plain text and validates if the software does what those specifications say, to produce causal model graphs for the Covasim, and with the result we can use to test the accuracy of the Covasim model.


\section{Overview of the Report}

Lorem ipsum dolor sit amet, consectetuer adipiscing elit. Aenean commodo ligula eget dolor. Aenean massa. Cum sociis natoque penatibus et magnis dis parturient montes, nascetur ridiculus mus. Donec quam felis, ultricies nec, pellentesque eu, pretium quis, sem. Nulla consequat massa quis enim. Donec pede justo, fringilla vel, aliquet nec, vulputate eget, arcu. In enim justo, rhoncus ut, imperdiet a, venenatis vitae, justo. Nullam dictum felis eu pede mollis pretium. Integer tincidunt. Cras dapibus. Vivamus elementum semper nisi. Aenean vulputate eleifend tellus. Aenean leo ligula, porttitor eu, consequat vitae, eleifend ac, enim. Aliquam lorem ante, dapibus in, viverra quis, feugiat a, tellus. Phasellus viverra nulla ut metus varius laoreet. Quisque rutrum. Aenean imperdiet. Etiam ultricies nisi vel augue. Curabitur ullamcorper ultricies nisi. Nam eget dui. Etiam rhoncus. Maecenas tempus, tellus eget condimentum rhoncus, sem quam semper libero, sit amet adipiscing sem neque sed ipsum. Nam quam nunc, blandit vel, luctus pulvinar, hendrerit id, lorem. Maecenas nec odio et ante tincidunt tempus. Donec vitae sapien ut libero venenatis faucibus. Nullam quis ante. Etiam sit amet orci eget eros faucibus tincidunt. Duis leo. Sed fringilla mauris sit amet nibh. Donec sodales sagittis magna. Sed consequat, leo eget bibendum sodales, augue velit cursus nunc.
