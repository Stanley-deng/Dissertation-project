\chapter{Introduction}

Throughout the years, companies have developed CPU with more and more powerful processing power, and these more powerful CPU comes the invention of supercomputer, these supercomputers have superior computational power compared to normal computers. With this computational power, people start to develop large and complex computational models. Computational model is a way to try and use computer system to simulate real life situation, these models use large amounts of parameters to determine the behavior or outcomes of a certain event, researchers can utilize these models to simulate experiments and real-life event \cite{Reference1} and aid their research with the results.\\*\\*
But these models aren’t without problems, one problem is these models has large number of inputs and outputs, and it takes a lot of times to process them, this means in order to test them, it also requires a lot of time to test all inputs and outputs’ accuracy with traditional method \cite{Reference2}. \\*\\*
There are several advanced testing techniques or tools that have been developed, but these techniques or tools usually require specific knowledge in software testing. The solution provided by this project is to implement a user interface, this interface minimizes the need for user to interact with the programming part of the testing process. Instead of require user to interact with the code directly, it visualizes the code into a way that is easy to read for user, highlight the parts that require edit and prompt what to edit for user.\\*\\
Causcumber is the tool this project mainly built upon. Causcumber is a tool that can determines the relationship between different input and output parameters and use this information to create graphs that focus on certain behaviours. Then just simply compare the difference between the test sets and the output of the tested models, people can know how accurate a certain interaction is between input and output parameters. With this method tester can saved a lot of time compared to traditional method where the best ways to test it is to run the model repeatedly with new data \cite{Reference3}. \\*\\*

The problem with this method is that there’s currently lack of a user-friendly way for people to access Causcumber, it requires a level of knowledge in software testing to understand how to operate this tool. With this interface, testing with Causcumber become more accessible for people who don’t have extensive knowledge in software testing.

\section{Aims and Objectives}

This project will be primary based on Causcumber, a tool for testing computational model and is mainly based on a model called Covasim, Covasim is a computational model focus on COVID-19 analyses, it can determine the infection rate and how will the rate change based on different interventions (such as quarantine, social distance, etc.) \cite{Reference4}. Causcumber is a in developing tool and is aimed to provide a testing method for computational model like Covasim, it uses Cucumber specification, another tool reads executable specification in plain text and validates if the software does what those specifications say, to produce casual model graph. \\*\\*
Since Causcumber is a new system that is still in development, there isn’t a convenient way for user to interact with the system. So, the primary goal of this project is to provide a user interface that can streamline the testing process, minimize the need to interact with programming code directly, allowed users to get a grasp of how to use this system more easily. \\*\\*
Another goal of this system is that it needs to be easy to read, and to achieve this, we use gherkin reference to help with it, by using a set of special keywords with a certain structure, it can give meaning to executable specifications. For the users who are not specialize in software engineering, this is a way to make the system easy to understand and can avoid a lot of confusion. \\*\\*



\section{Overview of the Report}

In the next chapter, Literature Survey, a list of literature will be providing information and reasons why there’s a need for this type of testing tool to exist. It will start with explaining what scientific software is, and what are their purpose. Follow up with why they are hard to test and why they aren’t usually tested in a proper method. Then provide some of the current testing methods and why those methods are insufficient. Then this will be followed up with introductions to Cucumber and Behave, two of the main factors of the Causcumber system. After that will be explain what Causal testing is and how computational model testing is could benefit from it. Finally, explaining the current flaw of this method and what can be done to improve it.\\* \\* 
In the third chapter, Requirements and Analysis, will be mainly talking about the requirements and objectives of the system. And what solution is possible based on these requirements\\* \\*
Part four, Design, will be discussing the design philosophy of the system, the design process and discuss what the system should do. \\* \\*
Part five, Implementation and testing, will be demonstrate the final system by introduce different parts of it and what functions do those parts have. \\* \\*
Part six, Evaluation, will provide two walkthrough of testing process with different models using the implemented user interface, each walkthrough will include the full testing process, the results, and a discuss about the result.
Part seven, Results and discussion, will discuss what does the results mean and how does the user interface help simplify the testing process, and what can be improve with the user interface.\\* \\*
Part eight, Conclusions, will summarize what has been achieved in this project, provide a overview on what the final product performs.






